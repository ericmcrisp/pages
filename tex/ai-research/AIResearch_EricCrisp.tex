%%%%%%%%%%%%%%%%%%%%%%%%%%%%%%%%%%%%%%%%%%%%%%%%%%%%%%%%%%%%%%%%%%%%%%%%
%%%%%%%%%%%%%%%%%%%%%% Simple LaTeX CV Template %%%%%%%%%%%%%%%%%%%%%%%%
%%%%%%%%%%%%%%%%%%%%%%%%%%%%%%%%%%%%%%%%%%%%%%%%%%%%%%%%%%%%%%%%%%%%%%%%

%%%%%%%%%%%%%%%%%%%%%%%%%%%%%%%%%%%%%%%%%%%%%%%%%%%%%%%%%%%%%%%%%%%%%%%%
%% NOTE: If you find that it says                                     %%
%%                                                                    %%
%%                           1 of ??                                  %%
%%                                                                    %%
%% at the bottom of your first page, this means that the AUX file     %%
%% was not available when you ran LaTeX on this source. Simply RERUN  %%
%% LaTeX to get the ``??'' replaced with the number of the last page  %%
%% of the document. The AUX file will be generated on the first run   %%
%% of LaTeX and used on the second run to fill in all of the          %%
%% references.                                                        %%
%%%%%%%%%%%%%%%%%%%%%%%%%%%%%%%%%%%%%%%%%%%%%%%%%%%%%%%%%%%%%%%%%%%%%%%%


%%%%%%%%%%%%%%%%%%%%%%%%%%%% Document Setup %%%%%%%%%%%%%%%%%%%%%%%%%%%%

% Don't like 10pt? Try 11pt or 12pt
\documentclass[10pt]{article}

% This is a helpful package that puts math inside length specifications
\usepackage{calc}
\usepackage[usenames,dvipsnames]{color}
\usepackage{multicol}

% Simpler bibsection for CV sections
% (thanks to natbib for inspiration)
\makeatletter
\newlength{\bibhang}
\setlength{\bibhang}{1em}
\newlength{\bibsep}
 {\@listi \global\bibsep\itemsep \global\advance\bibsep by\parsep}
\newenvironment{bibsection}%
        {\vspace{-\baselineskip}\begin{list}{}{%
       \setlength{\leftmargin}{\bibhang}%
       \setlength{\itemindent}{-\leftmargin}%
       \setlength{\itemsep}{\bibsep}%
       \setlength{\parsep}{\z@}%
        \setlength{\partopsep}{0pt}%
        \setlength{\topsep}{0pt}}}
        {\end{list}\vspace{-.6\baselineskip}}
\makeatother

% Layout: Puts the section titles on left side of page
\reversemarginpar

%
%         PAPER SIZE, PAGE NUMBER, AND DOCUMENT LAYOUT NOTES:
%
% The next \usepackage line changes the layout for CV style section
% headings as marginal notes. It also sets up the paper size as either
% letter or A4. By default, letter was used. If A4 paper is desired,
% comment out the letterpaper lines and uncomment the a4paper lines.
%
% As you can see, the margin widths and section title widths can be
% easily adjusted.
%
% ALSO: Notice that the includefoot option can be commented OUT in order
% to put the PAGE NUMBER *IN* the bottom margin. This will make the
% effective text area larger.
%
% IF YOU WISH TO REMOVE THE ``of LASTPAGE'' next to each page number,
% see the note about the +LP and -LP lines below. Comment out the +LP
% and uncomment the -LP.
%
% IF YOU WISH TO REMOVE PAGE NUMBERS, be sure that the includefoot line
% is uncommented and ALSO uncomment the \pagestyle{empty} a few lines
% below.
%

%% Use these lines for letter-sized paper
\usepackage[paper=letterpaper,
            %includefoot, % Uncomment to put page number above margin
            marginparwidth=.9in,     % Length of section titles
            marginparsep=.05in,       % Space between titles and text
            margin=0.5in,               % 1 inch margins
            includemp]{geometry}

%% Use these lines for A4-sized paper
%\usepackage[paper=a4paper,
%            %includefoot, % Uncomment to put page number above margin
%            marginparwidth=30.5mm,    % Length of section titles
%            marginparsep=1.5mm,       % Space between titles and text
%            margin=25mm,              % 25mm margins
%            includemp]{geometry}

%% More layout: Get rid of indenting throughout entire document
\setlength{\parindent}{0in}

%% This gives us fun enumeration environments. compactitem will be nice.
\usepackage{paralist}

%% Reference the last page in the page number
%
% NOTE: comment the +LP line and uncomment the -LP line to have page
%       numbers without the ``of ##'' last page reference)
%
% NOTE: uncomment the \pagestyle{empty} line to get rid of all page
%       numbers (make sure includefoot is commented out above)
%
\usepackage{fancyhdr,lastpage}
\pagestyle{fancy}
% \pagestyle{empty}      % Uncomment this to get rid of page numbers
\fancyhf{}\renewcommand{\headrulewidth}{0pt}
\fancyfootoffset{\marginparsep+\marginparwidth}
\newlength{\footpageshift}
\setlength{\footpageshift}
          {0.5\textwidth+0.5\marginparsep+0.5\marginparwidth-2in}
\lfoot{\hspace{\footpageshift}%
       \parbox{4in}{\, \hfill %
                    \arabic{page} of \protect\pageref*{LastPage} % +LP
%                    \arabic{page}                               % -LP
                    \hfill \,}}

% Finally, give us PDF bookmarks
\usepackage{color,hyperref}
\definecolor{darkblue}{rgb}{0.0,0.0,0.3}
\hypersetup{colorlinks,breaklinks,
            linkcolor=darkblue,urlcolor=darkblue,
            anchorcolor=darkblue,citecolor=darkblue}

%%%%%%%%%%%%%%%%%%%%%%%% End Document Setup %%%%%%%%%%%%%%%%%%%%%%%%%%%%


%%%%%%%%%%%%%%%%%%%%%%%%%%% Helper Commands %%%%%%%%%%%%%%%%%%%%%%%%%%%%

% The title (name) with a horizontal rule under it
%
% Usage: \makeheading{name}
%
% Place at top of document. It should be the first thing.
\newcommand{\makeheading}[1]%
        {\hspace*{-\marginparsep minus \marginparwidth}%
         \begin{minipage}[t]{\textwidth+\marginparwidth+\marginparsep}%
                {\LARGE \bfseries #1}\\[-0.8\baselineskip]%
                 \rule{\columnwidth}{0.25pt}%
         \end{minipage}}

% The section headings
%
% Usage: \section{section name}
%
% Follow this section IMMEDIATELY with the first line of the section
% text. Do not put whitespace in between. That is, do this:
%
%       \section{My Information}
%       Here is my information.
%
% and NOT this:
%
%       \section{My Information}
%
%       Here is my information.
%
% Otherwise the top of the section header will not line up with the top
% of the section. Of course, using a single comment character (%) on
% empty lines allows for the function of the first example with the
% readability of the second example.
\renewcommand{\section}[2]%
        {\pagebreak[3]\vspace{1.3\baselineskip}%
         \phantomsection\addcontentsline{toc}{section}{#1}%
         \hspace{0in}%
         \marginpar{
         \raggedright \small \scshape #1}#2}

% An itemize-style list with lots of space between items
\newenvironment{outerlist}[1][\enskip\textbullet]%
        {\begin{itemize}[#1]}{\end{itemize}%
         \vspace{-.6\baselineskip}}

% An environment IDENTICAL to outerlist that has better pre-list spacing
% when used as the first thing in a \section
\newenvironment{lonelist}[1][\enskip\textbullet]%
        {\vspace{-\baselineskip}\begin{list}{#1}{%
        \setlength{\partopsep}{0pt}%
        \setlength{\topsep}{0pt}}}
        {\end{list}\vspace{-.6\baselineskip}}

% An itemize-style list with little space between items
\newenvironment{innerlist}[1][\enskip\textbullet]%
        {\begin{compactitem}[#1]}{\end{compactitem}}

% An environment IDENTICAL to innerlist that has better pre-list spacing
% when used as the first thing in a \section
\newenvironment{loneinnerlist}[1][\enskip\textbullet]%
        {\vspace{-\baselineskip}\begin{compactitem}[#1]}
        {\end{compactitem}\vspace{-.6\baselineskip}}

% To add some paragraph space between lines.
% This also tells LaTeX to preferably break a page on one of these gaps
% if there is a needed pagebreak nearby.
\newcommand{\blankline}{\quad\pagebreak[2]\vspace{-0.3\baselineskip}}

% For \url{SOME_URL}, links SOME_URL to the url SOME_URL
\providecommand*\url[1]{\href{#1}{#1}}
% Same as above, but pretty-prints SOME_URL in teletype fixed-width font
\renewcommand*\url[1]{\href{https://#1}{\texttt{#1}}}

% For \email{ADDRESS}, links ADDRESS to the url mailto:ADDRESS
\providecommand*\email[1]{\href{mailto:#1}{#1}}
% Same as above, but pretty-prints ADDRESS in teletype fixed-width font
%\renewcommand*\email[1]{\href{mailto:#1}{\texttt{#1}}}

%%%%%%%%%%%%%%%%%%%%%%%% End Helper Commands %%%%%%%%%%%%%%%%%%%%%%%%%%%


%%%%%%%%%%%%%%%%%%%%%%%%% Begin CV Document %%%%%%%%%%%%%%%%%%%%%%%%%%%%

\begin{document}
\makeheading{Eric Crisp}

\section{Contact Information}
% NOTE: Mind where the & separators and \\ breaks are in the following
%       table.
%
% ALSO: \rcollength is the width of the right column of the table
%       (adjust it to your liking; default is 1.85in).
%
%\newlength{\rcollength}\setlength{\rcollength}{1.75in}
%
%\begin{tabular}[t]{@{}p{\textwidth-\rcollength-2.3in}p{2.3in}p{\rcollength}}
\begin{tabular}[t]{@{}p{4.2in}p{3.5in}}
\email{ecrisp@upenn.edu} & \url{ericmcrisp.github.io/pages} \\
(302) 528-2477 & \url{linkedin.com/in/ecrisp}
\end{tabular}

\section{Education}
  \textbf{University of Pennsylvania}, Philadelphia, PA
  \hfill Jan 2025 - Dec 2025 \\
  \-\quad \textbf{M.Sc., Data Science}

  \blankline

  \textbf{Pennsylvania State University}, State College, PA
  \hfill Aug 2015 - May 2021 \\
  \-\quad \textbf{M.Sc., Mechanical Engineering} \\ % \hfill GPA: 4.0
  \-\quad \textbf{B.Sc., Aerospace Engineering} % \hfill GPA: 3.6 

\section{Technical Skills}
    \begin{minipage}[t]{.3\textwidth}
    \textbf{Programming} \\
    \verb|Python, C++, MATLAB| \\
    \verb|JavaScript|
    \end{minipage}
    \begin{minipage}[t]{.4\textwidth}
    \textbf{Data Science, AI/ML} \\
    \verb|TensorFlow, PyTorch, Scikit-learn| \\
    \verb|SQL, Spark, Pandas, Numpy|
    \end{minipage}
    \begin{minipage}[t]{.3\textwidth}
    \textbf{Tools \& DevOps} \\
    \verb|Docker, AWS, CI| \\
    \verb|Git, React, Node|
    \end{minipage}

\section{Summary}
I am looking to transition into a role related to AI development. I am grateful to have had many opportunities throughout my career to develop engineering, communication, analytical skills along with leadership experience that blend well with the foundational AI/ML skills and knowledge developed at the Penn.

% \section{Papers}

% \textbf{<put a title here>}
% \hfill <put a publication date here> \\
% \textit{<put authors here>} \\
% \hfill \url{<put a link here>} \\
% put a summary here

% \blankline


\section{Experience}
\textbf{Lead Aerospace Engineer, Real-Time Modeling}
\hfill Apr 2022 -- Nov 2024 \\
\textbf{Blue Origin, Seattle, WA} \\
$\bullet$ Led a small, multi-disciplined team responsible for all RTM (real-time model) activities across Blue Origin. \\
$\bullet$ Developed RTMs for use in HIL, test support, controller development, and requirements validation including trade studies and performance optimization. \\
$\bullet$ Served as RTM project manager from project conception by managing scope, deliverables, and deligation. \\
$\bullet$ Identified critical software bugs on flight HIL systems via RTM integration, increasing reliability and value. \\
$\bullet$ Reduced testing manpower requirements by up to 35\% with RTM, accelerating development timelines. \\
$\bullet$ Effectively communicated the value and impact of technical outcomes from RTM to both technical and non-technical steakholders. \\
$\bullet$ Architected the RTM framework and developed source code, tooling, supporting algorithms and solvers.

\blankline

\textbf{Propulsion Development Engineer, Combustion Devices}
\hfill May 2021 -- Apr 2022 \\
\textbf{Firefly Aerospace, Austin, TX} \\
$\bullet$ Developed an automated thermal-structural design process that reduced engine production costs by 12\%. \\
$\bullet$ Contributed to clean sheet engine design through production, exceeding performance requirements by 4\%. \\
$\bullet$ Conducted root cause investigations of failures and implementated systematic and engineering solutions. \\ 
$\bullet$ Enhanced engine test visibility with automated visualizations of the engine state relative to test sequence.

% \section{Engineering Experience}

% \blankline

\section{Personal and Academic Projects}
\textbf{Fundamental Physics Models from Physics Informed Neural Networks}
\hfill Jul 2025 -- Present \\ 
$\bullet$ Investigating neural-symbolic approaches that combine Physics-Informed Neural Networks (PINNs) with transformer-based code generation models to model physical situations. \\
$\bullet$ Developing neural networks to automatically generate simulation code for simple physics problems, leveraging deep learning to bridge theoretical physics with computational implementation. \\
$\bullet$ Creating evaluation framework to identify where AI-generated simulations violate fundamental conservation laws (energy, mass, momentum), providing insights into model limitations in implementing within scientific computing domains.

\blankline

\textbf{Machine Learning Pipeline for Food Classification and Health Scoring}
\hfill May 2025 -- Jul 2025 \\ 
$\bullet$ Built an end-to-end ML pipeline to classify food items and generate health scores using supervised learning algorithms, with model optimization through GridSearchCV hyperparameter tuning achieving 91\% accuracy on test data. \\
$\bullet$ Implemented comprehensive data preprocessing using Pandas for large-scale dataset manipulation, NLP techniques for ingredient text processing and nutritional analysis, normalization, imputing, and encoding for PCA analysis and created visualizations with Seaborn and Matplotlib to present process results.

% \blankline

% \textbf{Full Stack Reccomendation System}
% \hfill May 2025 -- Jul 2025 \\ 
% $\bullet$ Built a full-stack application using AWS RDS, React, Node.js, and NLP to help users identify restaurants in their city, discover similar options, and receive personalized meal and restaurant suggestions. \\
% $\bullet$ Processed and integrated large-scale datasets (Yelp, recipe data, Kaggle sources) containing millions of records into a PostgreSQL database on AWS RDS, implementing optimized queries and RESTful APIs to serve real-time recommendations to users.

% \blankline

% \textbf{Ground Up Machine Learning and Neural Network Development}
% \hfill Jan 2025 -- May 2025 \\ 
% $\bullet$ Implemented various machine learning algorithms (PCA, SVM, K-means, linear and logistic regression, gradient descent, lasso, ridge, and net elesatic regression) and simple neural network from scratch.

% \blankline

% \section{Open Source Contributions} 
%   \textbf{<title>}
%   <description>
%   \hfill <date>

% \section{Relevant Courses}
% \textbf{Statistics, Analysis of Algorithms, Linear Algebra and Optimization} \hfill Spring 2025 \\
% \textbf{Artificial Intelligence, Computer Systems, Big Data Analytics, Databases} \hfill Summer 2025 \\
% \textbf{Machine Learning, Internet and Web Systems, Deep Learning} \hfill  \textit{Fall 2025}


\section{Awards and Activities}
  \textbf{Blue Origin Engines Challenge Award}
  \hfill Jul 2022 \\
  Awarded for technical successes in developing the real-time modeling capabilities at Blue Origin.

%   \textbf{Blue Origin Liftoff Award}
%   \hfill Jan 2023 \\
%   Nominated by peers and team members for leadership, technical excellence, and having a bias for action.


% \blankline

%   Harvey Mudd Effective Altruism Club Leader (2017 -- 2019) --- AI Summer Fellows Program Attendee (2018, 2019) --- Effective Altruism Global Attendee (2017, 2019, 2020) --- World Wide Web Consortium Interledger Payments Community Group Member (2016) --- National Forensics League Honor Society Outstanding Distinction (2015) --- National Policy Debate Tournament of Champions Participant (2014, 2015)

%\section{Project Experience}
    %\textbf{Project Title}
    %\hfill \textbf{Dates -- You know the drill} \\
    %\textit{Where you were when you did the project} \\
    %Some sort of description goes here, preferably including
    %\href{google.com}{links} of some sort.

% \section{References}
% \begin{tabular}[t]{@{}p{3.0in}p{3.0in}}
% \textbf{Robert St. Clair}     & \textbf{Robert Aguilar}  \\
% Sr. Manager: Engine Systems Analysis        & Principal Engineer: Real Time Modeling  \\                           \email{rstclair@blueorigin.com}        & \email{raguilar@blueorigin.com}     \\
% \end{tabular}

\end{document}

%%%%%%%%%%%%%%%%%%%%%%%%%% End CV Document %%%%%%%%%%%%%%%%%%%%%%%%%%%%%

%----------------------------------------------------------------------%
% The following is copyright and licensing information for
% redistribution of this LaTeX source code; it also includes a liability
% statement. If this source code is not being redistributed to others,
% it may be omitted. It has no effect on the function of the above code.
%----------------------------------------------------------------------%
% Copyright (c) 2007, 2008, 2009, 2010, 2011 by Theodore P. Pavlic
% Some modifications made by Aaron Gable, 2011.
% More modifications made my John Phillpot, 2015.
% More modifications made my Evan Hubinger, 2016.
% More modifications made my Eric Crisp, 2025.
%
% Unless otherwise expressly stated, this work is licensed under the
% Creative Commons Attribution-Noncommercial 3.0 United States License. To
% view a copy of this license, visit
% creativecommons.org/licenses/by-nc/3.0/us/ or send a letter to
% Creative Commons, 171 Second Street, Suite 300, San Francisco,
% California, 94105, USA.
%
% THE SOFTWARE IS PROVIDED "AS IS", WITHOUT WARRANTY OF ANY KIND, EXPRESS
% OR IMPLIED, INCLUDING BUT NOT LIMITED TO THE WARRANTIES OF
% MERCHANTABILITY, FITNESS FOR A PARTICULAR PURPOSE AND NONINFRINGEMENT.
% IN NO EVENT SHALL THE AUTHORS OR COPYRIGHT HOLDERS BE LIABLE FOR ANY
% CLAIM, DAMAGES OR OTHER LIABILITY, WHETHER IN AN ACTION OF CONTRACT,
% TORT OR OTHERWISE, ARISING FROM, OUT OF OR IN CONNECTION WITH THE
% SOFTWARE OR THE USE OR OTHER DEALINGS IN THE SOFTWARE.
%----------------------------------------------------------------------%